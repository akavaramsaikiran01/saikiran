
\documentclass{knittingpattern}


\definecolor{colour0}{HTML}{000000}
\definecolor{colour1}{HTML}{EDB668}
\definecolor{colour2}{HTML}{000000}
\definecolor{colour3}{HTML}{d49fce}
\definecolor{colour4}{HTML}{a377d1}
\definecolor{colour5}{HTML}{70de6a}
\definecolor{colour6}{HTML}{FFFF00}
\definecolor{colour7}{HTML}{f2b530}
\definecolor{colour8}{HTML}{70de6a}
\usepackage{graphicx}
\usepackage{wrapfig}
\usepackage[document]{ragged2e}
\begin{document}
%\usepackage{chemfig}

\title{ASSIGNMENT 1}
\author{\huge \textbf {A.SAI KIRAN ,  ID: B192662}}
\maketitle
\section{KNITTING PATTERNS :}
  \large This class provides a very convenient way to introduce boxed diagrams.
 we are thus going to use our stock image a few featuresto make knitting instructions more redable , hpwever,we can adapt them to make prettier documents for our purpose as well.
  \begin{figure}[h]
      \centering
      \includegraphics[width=10cm]{img.jpg}
      
  \end{figure}
\note{colour0}{colour7}{}{ we have a of highlighting important text , or as was originally indented,important \\instructions.Feel free to choose whatever background and border colour you like when you replicate these features , but try to replicate dimensions as well as you can.}
\begin{pattern}{colour4}{colour5}
COURSE & CREDITS\\
Intriduction to computer programming & 6\\
Abstractions and paradigms in programming & 6\\
Abstractions and paradigms in programming lab & 3\\
Data structures and algorithms & 6\\
software system lab & 8
\end{pattern}
\note{colour2}{colour3}{NOTE}{This is a note.The feature was introduced to typeset a sequence of knitting instructions .This first coloumn is for the instruction,the second for the number of stitche.But they ,it looks cool !}
\newpage

  \begin{wrapfigure}{l}{0.5\textwidth} 
    \centering
    \includegraphics[width=0.4\textwidth]{img.jpg}
\end{wrapfigure}

\large Onhd dnjnje njksja nvmkd eikslopa bvnmsksid njvkoslp fhsjkmnck mnjckd kfmnj jskjmn bmksj dkkkowlpakd nbvjkaolwp mnckopwkkdf njkmnalolpq jjje mndjldmd mjfjhuyrjkhd fjhfuih left.

\large Onhd dnjnje njksja nvmkd eikslopa bvnmsksid njvkoslp fhsjkmnck mnjckd kfmnj jskjmn bmksj dkkkowlpakd nbvjkaolwp mnckopwkkdf njkmnalolpq jjje mndjldmd mjfjhuyrjkhd fjhfuih left.
\end{document}

\documentclass{knittingpattern}


\definecolor{colour0}{HTML}{000000}
\definecolor{colour1}{HTML}{EDB668}
\definecolor{colour2}{HTML}{000000}
\definecolor{colour3}{HTML}{d49fce}
\definecolor{colour4}{HTML}{a377d1}
\definecolor{colour5}{HTML}{70de6a}
\definecolor{colour6}{HTML}{FFFF00}
\definecolor{colour7}{HTML}{f2b530}
\definecolor{colour8}{HTML}{70de6a}
\usepackage{graphicx}
\usepackage{wrapfig}
\usepackage[document]{ragged2e}
\begin{document}
%\usepackage{chemfig}

\title{ASSIGNMENT 1}
\author{\huge \textbf {K.SHIVA SAI RAM ,  ID: B191493}}
\maketitle
\section{KNITTING PATTERNS :}
  \large This class provides a very convenient way to introduce boxed diagrams.
 we are thus going to use our stock image a few featuresto make knitting instructions more redable , hpwever,we can adapt them to make prettier documents for our purpose as well.
  \begin{figure}[h]
      \centering
      \includegraphics[width=10cm]{img.jpg}
      
  \end{figure}
\note{colour0}{colour7}{}{ we have a of highlighting important text , or as was originally indented,important \\instructions.Feel free to choose whatever background and border colour you like when you replicate these features , but try to replicate dimensions as well as you can.}
\begin{pattern}{colour4}{colour5}
COURSE & CREDITS\\
Intriduction to computer programming & 6\\
Abstractions and paradigms in programming & 6\\
Abstractions and paradigms in programming lab & 3\\
Data structures and algorithms & 6\\
software system lab & 8
\end{pattern}
\note{colour2}{colour3}{NOTE}{This is a note.The feature was introduced to typeset a sequence of knitting instructions .This first coloumn is for the instruction,the second for the number of stitche.But they ,it looks cool !}
\newpage

  \begin{wrapfigure}{l}{0.5\textwidth} 
    \centering
    \includegraphics[width=0.4\textwidth]{img.jpg}
\end{wrapfigure}

\large Onhd dnjnje njksja nvmkd eikslopa bvnmsksid njvkoslp fhsjkmnck mnjckd kfmnj jskjmn bmksj dkkkowlpakd nbvjkaolwp mnckopwkkdf njkmnalolpq jjje mndjldmd mjfjhuyrjkhd fjhfuih left.

\large Onhd dnjnje njksja nvmkd eikslopa bvnmsksid njvkoslp fhsjkmnck mnjckd kfmnj jskjmn bmksj dkkkowlpakd nbvjkaolwp mnckopwkkdf njkmnalolpq jjje mndjldmd mjfjhuyrjkhd fjhfuih left.
\end{document}
